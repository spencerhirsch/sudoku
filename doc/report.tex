\documentclass{article}
\usepackage{hyperref}
\usepackage[margin=1in]{geometry}
\usepackage{indentfirst}   % Indents first paragraph. change if u want ig
\usepackage{setspace} 
\doublespacing
\usepackage{graphicx}
\graphicspath{ {./output_images/} }
\usepackage{listings}
\usepackage{xcolor}
\lstset{
  basicstyle=\ttfamily,
  columns=fullflexible,
  breaklines=true,
  postbreak=\raisebox{0ex}[0ex][0ex]{\color{red}$\hookrightarrow$\space}
}

\begin{document}
\title{\textbf{Intermediate Submission Questions}}
\author{Spencer Hirsch, Thomas Johnson}
\date{\today}

\maketitle

\textbf{Psuedo-Code for Backtracking Algorithm:}

\medskip

\begin{lstlisting}[frame=single]
    useful funcs to make:
    valid_move_list() -> returns a list of valid numbers a cell can take
        check rows, and columns for unused numbers

    solve(board)        true or false
        if board if full
            save board
            return true

        for open cell in board
            nums = valid_move_list(cell)
            for each num in nums
                if cell can be filled with num
                    fill it
                    if (solve(board))
                        return true   / filled up board with valid moves

                mark cell empty / backtracks

            return false / the open cell didnt have a valid number to put, making this patheway unsolveable
\end{lstlisting}

\pagebreak


\textbf{Screenshot demonstrating compilation of code:}

\bigskip

\includegraphics{compilation.png}

\textit{Screenshot demonstrating that the code runs.}

\pagebreak

\textbf{Screenshot showing output for first test case:}

\bigskip

\includegraphics{test1.png}

\textit{Screenshot demonstrating our first test case, as well as correct output.}

\pagebreak

\textbf{Screenshot showing output for second test case:}

\bigskip

\includegraphics{test2.png}

\textit{Screenshot demonstrating our second test case, as well as the correct output.}

\pagebreak

\textbf{Screenshot showing output for third test case (impossible):}

\bigskip

\includegraphics{impossible_output.png}

\textit{Screenshot demonstrating the output for an impossible puzzle given to the program.}

\pagebreak

\textbf{Screenshot showing output for fourth test case:}

\bigskip

\includegraphics{test4.png}

\textit{Screenshot demonstrating the output for a fourth test case, for good measure.}

\pagebreak

\textbf{Summary of the intermediate submission:}

\bigskip

\noindent For this assignment, my partner and I used the backtracking algorithm in order
to solve for the problem. Our program reads in a text file that conatins the 
number of rows and columns as well as a list of the values that the matrix will
be made up of. The values will be read in by row,

\[R_{00},R_{01},R_{02},R_{03},R_{10},R_{11},R_{12},R_{13},R_{20},R_{21},R_{22},R_{23},R_{30},R_{31},R_{32},R_{33}\]

\noindent The values are then placed in their repsective places in the n $\times$ n matrix.
The initial empty spaces hold a value of 0, the algorithm will search for the 0's 
in the matrix and replace them with their correct value. We chose to use a backtracking
algorithm in order to solve this problem. The psuedo-code for our solution is posted above.
The solution that we came to is our original code.

\end{document}